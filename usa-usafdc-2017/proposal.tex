%%% Please see https://math.mit.edu/services/grants/index.php for information and instructions
%%%%%%%%% MASTER -- compiles the 4 sections

\documentclass[10pt]{article}


\usepackage[letterpaper, margin=1in]{geometry}



\usepackage{fancyhdr}

\pagestyle{fancy}
\fancyhf{}
\chead{USAFDC GRANT APPLICATION}
\cfoot{\thepage}

\parskip0.5em

%PUT YOUR MACROS HERE

\usepackage{amsmath}
\usepackage{amssymb}

\usepackage[hidelinks]{hyperref}

\bibliographystyle{plain}

\begin{document}

% \begin{center}
%   \textbf{
%   UNIVERSITY OF SOUTH ALABAMA FACULTY DEVELOPMENT COUNCIL
%   (USAFDC) PROPOSALS
%   }
% \end{center}
%
% INSTRUCTIONS:  Please refer to the General Guidelines for information on preparing your proposal.
% Submit your completed proposal signed by you, your department chair, and dean to the Graduate School Office, AD 300.
%
%
% \section{PROPOSAL IDENTIFICATION}
%
% \begin{center}
% \begin{tabular}{ccc}
%   & \textbf{NAME} & \textbf{DEPARTMENT} \\
% \textbf{Principal Investigator (PI)} &
% \underline{Steven Clontz} &
% \underline{Mathematics and Statistics}
% \end{tabular}
% \end{center}

% A. Title of Proposal:
%
% \hspace{3em}
% \underline
% {Enhancing Research and Cyberinfrastructure in Set-Theoretic Topology}
%
% B. Project Period:
%
% \hspace{3em}
% FROM: \underline{2017 May 1}
% \hspace{2em}
% TO: \underline{2018 August 31}
%
% C. Amount Requested:
%
% \hspace{3em}
% Total Project Cost:
% \underline{\$0}
%
% \hspace{3em}
% Other Sources:
% \underline{\$0}
%
% \hspace{3em}
% \textbf{Requested from USAFDC:
% \underline{\$0}}
%
%
% \newpage
%
% \section{Principal Investigator (PI):}
%
% Agrees to accept responsibility for the scientific and technical conduct of the proposed project, for provision of the required progress and final reports, and will obtain approval of the appropriate committee, as indicated below, and adhere to applicable regulations for projects involving any of the following:
%
% \begin{center}\footnotesize
% \begin{tabular}{|c|l|c|c|c|}
%   \hline
%   && Needed & Submitted & Approved \\\hline
%   A. & Radiation Safety & N/A && \\\hline
%   B. & Human Subjects & N/A && \\\hline
%   C. & Animal Use & N/A && \\\hline
%   D. & Bio Hazardous Matierals & N/A && \\\hline
%   E. & Export Controls & N/A && \\\hline
%   F. & Other Environmental or Chemical Hazards (Describe Below) & N/A && \\\hline
% \end{tabular}
% \end{center}
%
% Contact Dusty Layton, Director Research Compliance Assurance, CSAB 128, with any questions on items A-F.
%
% Note: I understand that if my proposal is recommended for funding, the award will be conditional until the appropriate approvals above have been received and provided by me to the Graduate School Office.

\setcounter{page}{3}

\centerline{\bf\Large
ABSTRACT
}

\textit{Abstract your proposed project (500-word limit) below using non-technical language easily understandable to persons outside your field.  Remember that a diverse group of faculty will be reviewing your proposal.}\vspace{2em}

The Encyclopedic Database of Topological Spaces, also known as the \(\pi\)-Base and currently hosted at http://topology.jdabbs.com, is an open-source web application with the ability to cross-reference topological spaces, their properties, and the theorems which connect them. As an example, once a user has logged that the closed interval of real numbers between 0 and 1 is ``compact'' and ``Hausdorff'', and EDTS automatically deduces that the space is also ``normal'', and includes this space in search results for normal topological spaces. Historically, researchers memorized hundreds of various spaces, properties, and theorems from papers spread across numerous journal articles, but with our body of knowledge expanding exponentially as time progresses, a robust EDTS would facilitate efficient research by eliminating the difficulty of memorizing or finding such results in the ever-growing and ever-fragmenting literature. For comparison, the similar and more mature On-Line Encyclopedia of Integer Sequences, launched in its current form in late 2010, has been cited in over 5002 works. The applicant will use support from this award to collaborate with fellow researcher Ziqin Feng (Auburn University) and EDTS's main programmer James Dabbs to polish the EDTS prototype for use in production by active researchers in topology, with the assistance of several active researchers who will beta test the EDTS platform throughout the project. Ultimately, the investigator aims to establish EDTS as a leading tool in researching and cataloging topological spaces and objects in other mathematical categories; by developing this prototype to a minimally viable product for active use by researchers, the investigator’s competitiveness for external funding to support the project will be increased.

\newpage

\centerline{\bf\Large
DETAILED BUDGET
}

\begin{center}
\begin{tabular}{|l|c|c|c|}
  \hline
  & USAFDC & Other Sources & Total Anticipated Budget \\\hline
  Books &&& \\\hline
  Copying, Printing &&& \\\hline
  Equipment &&& \\\hline
  Postage &&& \\\hline
  Professional Fees** &&& \\\hline
  Software & \$100 && \$100 \\\hline
  \parbox{1.5in}{Student Wages*\\ at \$17.50 per Hr} & \$2100 && \$2100 \\\hline
  Supplies, Consumables &&& \\\hline
  Telephone &&& \\\hline
  Travel & \$2090 && \$2090 \\\hline
  Other &&& \\\hline\hline
  \textbf{TOTAL} & \textbf{\$4290} && \textbf{\$4290} \\\hline
\end{tabular}
\end{center}

* In the budget justification respond to Item 10 of general guidelines.

** Consulting or other professional services that are not available through the University.

\newpage




\centerline{\bf\Large
BUDGET JUSTIFICATION (Required)
}

\textit{Provide a line-by-line justification for each item in the budget, including details of calculated totals as well as how each is necessary for execution of the project.  Itemize equipment and detail travel. For proposals that will take longer than one year to accomplish, the budget and all project plans should fully reflect the timeline required.  Use additional pages as necessary.}\vspace{2em}

\textbf{Software:} \textbf{\$120} is requested to cover hosting costs for the EDTS website on a Digital Ocean cloud server for two years. \textbf{\$80} is requested to obtain the domain name edts.io (or similar) for two years.

\textbf{Student Wages:} One graduate student will be hired for Summer 2018 to assist the investigator with the auditing and expansion of the EDTS database. This student will be compensated at \$17.50 per hour (as limited by the proposal preparation guidelines), 15 hours per week, for 8 weeks, totaling \textbf{\$2100}. The work done by this student supports faculty research by filtering through the trivial (but time-consuming) data entry tasks, freeing the investigator to answer the non-trivial questions associated with this data entry (see the Methodology section for more details).

\textbf{Travel:} The investigator will travel to Auburn, AL for three days during May 2017 to meet with collaborator Prof. Ziqin Feng of Auburn University to develop a plan for auditing the EDTS database for use in cutting-edge research and marketing the tool to the community of researchers, including a selection of appropriate candidates for using the EDTS beta platfrom to assist their research. Compensation for mileage is estimated at \textbf{\$250}, and six days of meals is estimated at \textbf{\$165}. (Accommodations will be provided by Dr. Feng.)

The investigator will host the EDTS developer James Dabbs for one week during Summer 2017 to collaborate on the EDTS source code, making the necessary backend and user interface changes to ready it for production on a new domain and server for active use by researchers in topology. Airfare from Washington D.C. to Mobile is estimated at \textbf{\$700}, and five days of meals is estimated at \textbf{\$275}. Hotel accommodations for six nights are estimated at \textbf{\$700}.

\newpage




\centerline{\bf\Large
Proposed Work Plan
}

\textit{Complete the following sections. Adjust the space provided as necessary but the work plan may not exceed five pages, excluding works cited.  Supporting material may be attached as appendices but this is not necessarily encouraged.  Please be reminded that the work plan should be easily understood by a broad academic audience.}


\section*{A. Introduction}

\subsection*{1. State objective(s) of proposed work.}

The main objective of this project is to mature the \(\pi\)-Base web application
for cataloging topological spaces, theorems, and properites from a casual
side-project into a robust research tool for mathematicians in the fields of
general and set-theoretic topology, an Encylopedic Database of Topological
Spaces (EDTS). The current application, hosted at
http://topology.jdabbs.com/, is generally recieved well as a novelty by
researchers, but has several flaws preventing it from being used as a
reference in serious research.

The most glaring of these is a lack of
peer-review; the \(\pi\)-Base database is currently populated by a mix of
results from
the literature and unverified data from unvetted users. This will be
corrected by auditing the contents of the database to reference peer-reviewed
sources where possible, and flagging other entries as unverified.
In addition, several important data points from the literature are not
represented in the database, leaving too many holes for it to serve as
an efficient resource resource for most researchers.
To that end, the database will be expanded to a critical mass that meets the
needs of several specific researchers in several niches of set-theoretic
topology, who will be recruited by the investigator and his colleague Prof.
Ziqin Feng of Auburn University. To support these necessary enhancements,
the investigator will collaborate with \(\pi\)-Base developer James Dabbs to
write the code required for these new features and user interface improvements.

In addition to revolutionizing the way research in set-theoretic and general
topology is done, an additional objective of the project is to develop
a mechanism for finding suitable undergraduate research projects in those
fields. Research in mathematics typically involves finding logical proofs or
counter-examples to mathematical conjectures which have yet to be solved. Due
to this, it is often difficult to find
accessible problems for undergraduates which would constitute
truly original research. Most problems on the mind of topological researchers
that could be tackled by undergraduates
are usually solved trivially or at least in short time by mathematicians with
more experience. However, once the EDTS database reflects the cutting-edge
of topological research, it will be able to
automatically produce open questions that are
truly unknown, yet have not been considered by PhD researchers. Of course,
what makes a question in topology ``interesting'' to researchers is determined
by humans working in the field, so the majority of these
computer-generated open questions
would not be attractive to more advanced researchers, even those that are
easily seen to not be difficult to solve. Such questions
would be perfect for
creating undergraduate projects or theses that allow students to work on
truly original research in a field traditionally inaccessible to students
without graduate cowasursework in the area.

Another objective for this project is to pursue external funding for the
continued development of the EDTS platform and database, as described below
under Anticipated Outcomes.





\subsection*{2. Background:  Review scholarly work in the area and its relationship to the proposed study.}

In 1970, Steen and Seebach published a handbook
covering \(143\) important examples of topological spaces, entitled
\textit{Counterexamples in Topology} \cite{MR1382863}. Renowned researcher
Mary Ellen Rudin wrote the following in her review \cite{MR1536430}
of the original text.
``\textit{Counterexamples in Topology} is a valuable addition to the small
collection of books I keep on the shelf in my office.'' ``The book is
completely unique; no other book now in print serves its purpose.''
Recognizing the maze of counterexamples littered throughout the
field of set-theoretic topology, Rudin suggested that not only students could
benefit from the text as a guidebook, but
``Even those of us who work exactly in the area will profit from
its organization.''

Several other \textit{Counterexamples} texts
in the tradition of Stein and Seebach have been published in several fields,
including (but not limited to) real analysis \cite{MR1256489}, differential
equations \cite{MR1113487}, probability \cite{MR930671}, and
graph theory \cite{MR0491272}.
The utility of such books is evident to the working mathematician, as it is
invaluable to have a convenient resource that efficently
answers questions of the following forms:
\begin{itemize}
  \item ``Does the object \(T\) have property \(X\)?''
  \item ``Do all objects with properties \(\{A,B,C\}\) also have properties
          \(\{X,Y,Z\}\)?''
  \item ``Do we already know of an object that satisifes properties
          \(\{A,B,C\}\)?''
\end{itemize}

The limitations of these publications are just as evident. When entered into
the existing \(\pi\)-Base prototype and automatically checked by the computer,
several non-obvious errors were unearthed in
\textit{Counterexamples in Topology}. This is to be
expected in such an ambitious work, but certainly would be a frustrating
discovery for a mathematican relying on the faulty data. A computer-verified
resource such as EDTS would prevent such inconsistencies from ever appearing
in the database.

Furthermore, over \(45\) years later, \textit{Counterexamples in Topology}
is no longer on the cutting edge of research. It was originally written with
the assistance of an NSF-funded undergraduate research experience, and
represented the most recent developments of the time. However, as a book,
it is not a living document and cannot evolve with the active community of
research to continually serve its needs. The EDTS will continually be updated
and expanded, allowing it to serve researchers of topology for years to come.
In addition, the software that will power EDTS will be written in a generic
fashion, so that it may similarly be used to power databases for literally
any category of mathematics, including those with existing books and databases
of counterexamples, and those which as of yet do not.

The benefits of such a robust database are already known in another field of
mathematics. The On-Line Encyclopedia of Integer Sequences, hosted at
https://oeis.org, is a frequently used and cited resource in number theory
and its related fields.
Since the site launched in its current form in late 2010, the OEIS
has been cited in over 5002 works \cite{OEIScitations}. The success of OEIS
is due to its ease of access and use by both researchers and students of
mathematics, its robust referencing system connecting the database to the
literature, and the community that maintains it. It is the investigator's
expectation that an enhanced Encylopedic Database of Topological Spaces can
acheive a similar level of success and notability by following the example
of OEIS.







\subsection*{3. Preliminary work:  Describe any work you have begun that relates to the project and how the proposed work validates or extends that work.}

The \(\pi\)-Base protoype currently hosted at http://topology.jdabbs.org
represents the preliminary work done for this project by the investigator,
its developer, and the small community that has contributed to it
over the past few years of its existence (similar to Wikipedia, but without
the sizable userbase of content experts). Specifically, the investigator
has contributed to the project by entering data related to his own research
in set theoretic topology, for example
\cite{MR3467819,MR3482726,MR3227201,MR3438747}. Using this prototype has
allowed the investigator to cleanly catalog the raw data associated with
the results of his research and generate open questions that have yet to be
solved by those results; for example, despite the theorem
http://topology.jdabbs.com/theorems/168
(discussed in \cite{clontzMengerGamePreprint}) which states that all
\(2\)-Markov-Winning Menger spaces are Winning Menger, there is no known
counter-example to the converse: a Winning Menger space which isn't also
\(2\)-Markov-Winning Menger.

Despite its promise, the utility of the existing prototype is limited by the
current functionality of the site and the limited involvement by the broader
community of researchers. The proposed work will allow the prototype to
mature into a robust web applicaiton that immediately communicates its
utility to acitve researchers. Combined with the proposed improvments to the
\(\pi\)-Base software and user interface, the recruitment of leading
researchers to contribute and maintain the database by the investigator and
Dr. Feng will establish the new EDTS application as a primary resource for
researchers to find and cite research results in topology. After its
success is established, the EDTS software can then be expanded to serve other
fields of mathematical research as well.











\subsection*{4. Significance:  State the potential importance of the proposed work to the field of study.  Also, please attach a letter of support from your department chair (or equivalent position) evaluating the merit of the proposal.}

The EDTS will revolutionize the way research is done in general and
set-theoretic topology. The density of modern literature in our field is
a huge roadblock for efficient work in advancing research; without an efficient
mechanism for sharing the results of our research (besides the consumption
and memorization of thousands of pages of an ever-increasing body of
literature), it is quite common for
researchers to waste non-trivial amounts of time working on problems that
have already been solved in a paper they had been unaware of. Other ``problems''
often can be solved with the right combination of results from two previously
unconnected papers. Making these connections can be difficult, especially
when the papers come from different niches of research. However, this
difficulty is non-existent for a software package that automatically catalogs
and connects entries to its database made by its users.

The benefits EDTS will have for undergraduate research (as described above
in the stated objectives for this work) will also encourage more young
researchers to enter the field of set-theoretic topology. Many
mathematics graduate students have limited experience with these fields, but
with an improved mechanism to develop appropriate undergraduate research
projects, more students will view this field as more accessible and consider
pursuing related research at the masters or doctoral level.












\section*{B. Methodology}

\subsection*{1.	Describe in detail your proposed work or approach to accomplishing the objective(s).}


In May 2017 the investigator will travel to Auburn University
  for three days to
  collaborate with Ziqin Feng to identify the changes necessary
  to transform the existing
  \(\pi\)-Base prototype into an appropriate platform for
  active use by researchers. Several
  candidates who will be recruited to use and contribute to
  beta version of the EDTS platform will be identified. Other topology
  researchers at Auburn, including
  Michel Smith, Stuart Baldwin, and Gary Gruenhage, will be consulted to
  provide further guidance and advice on what other improvements should be
  made to the prototype.

Later that summer, \(\pi\)-Base developer James Dabbs will travel from
  Washington, D.C. to Mobile for one week
  to collaborate with the investigator on
  the development of the \(\pi\)-Base source code, transforming the
  prototype into a useable beta version of the future EDTS platform.
  A work plan will be made
  for the continued development of the EDTS codebase through
  the end of the following summer. Following this initial work of
  software development, the investigator and Dabbs will
  collaborate on this source code development remotely according to
  this plan.

Also during that summer, the chosen candidates for beta-testing
  will be recruited.
  Throughout the fall and spring semesters of the 2017-18 academic year,
  these testers will use
  the EDTS beta version to assist with their active research.
  Holes in the database will be identified, along with corresponding
  sources from the literature that can fill those holes: textbooks that
  provide basic facts, and journal articles that build up to the areas
  of active research studied by the participating beta-testers.

Finally in Summer 2018, a graduate student in mathematics will be hired to
  assist the investigator in entering the data from these identified sources
  into the EDTS database, with the consultation of the beta-testers to
  ensure the expanded database is sufficient to support
  their regular research. Much of this work is trivial, consisting of
  taking results from these papers and entering them into EDTS in an appropriate
  format. However, due to inconsistencies between authors or the use of
  non-standard terminology or methodologies, the student will not be able to
  complete this work alone. These issues will be filtered out by the student
  for consideration by the investigator, allowing him to handle the
  non-trivial issues that arise in serializing mathematical results that
  were not originally written with such serialization in mind.

At the conclusion of the project in August 2018, EDTS will be officially
released to the research community as version 1.0. The investigator and
his collaborators will promote the new platform in appropriate venues,
including professional conferences and scholarly articles.













\subsection*{2. If data are to be collected, outline the means by which they will be analyzed.}

N/A












\subsection*{3. Timeline.  Indicate the proposed timeline for project completion. For projects exceeding one year (12 calendar months), provide justification.}


\begin{itemize}
  \item May 2017. Investigator travels to Auburn University for preliminary
    consultations with colleagues, including Dr. Ziqin Feng.
  \item Summer 2017. \(\pi\)-Base developer James Dabbs travels to Mobile
    for one week to collaboraite with the investigator on initial development
    of EDTS codebase from exisiting prototype. Testers are recruited
    for EDTS beta.
  \item Sepetember 2017: Beta-testing of EDTS begins. AMS Southeastern Meeting
    for Fall 2017 held in Orlando, FL.
  \item April 2018: AMS Southeastern Meeting for Spring 2018 held in Nashville,
    TN.
  \item Spring 2018: 52nd Spring Topology and Dynamics Conference held at TBD
    location.
  \item June-July 2018: Student hired to assist with data entry tasks based on
    beta-testing feedback and suggestions.
  \item August 2018: EDTS version 1.0 released.
\end{itemize}

The project timeline extends a total of sixteen months. Ideally, the student
worker would be available to contribute to the project earlier; however,
the appropriate coursework that enables a student to contribute to this project
(MA 434 or MA 542) will not be offered until Fall 2017. Due to the usual
workload of graduate assistants in the department, it would be infeasible to
begin work on the project during the following spring, leaving Summer 2018
as the ideal time for the student's contributions. In addition, the start of
the project is best scheduled for May 2017, allowing for more flexible travel
by the investigator and collaborator James Dabbs during the May and Summer
semesters. This additional time will also facilite the careful feedback of
testers and selection of appropriate literature to ensure that the EDTS
platform meets the needs of active researchers. Support for the minimal
costs to host the EDTS platform for two total years is requested so that
the results of the project may continue to serve the research community
while external funding is sought.










\section*{C. Anticipated outcomes:}

\subsection*{1. Include names of probable journals, publishers, etc.  A final written report and a presentation of project results at the Annual Research Forum are required; failure to comply may jeopardize future USAFDC funding opportunities.}

Typical journals publishing in the area to be served by EDTS include
  \textit{Topology and Its Applications},
  \textit{Topology Proceedings},
  \textit{Rocky Mountain Journal of Mathematics},
  \textit{Commentationes Mathematicae Universitatis Carolinae},
  \textit{Questions and Answers in General Topology},
  \textit{Fundamenta Mathematicae},
  \textit{Houston Journal of Mathematics}, and many others.
  A survey of the work associated with this project will be submitted
  to \textit{Topology Proceedings}, due to its relationship with the Spring
  Topology and Dynamaics Conference and Summer Conference on Topology and
  Its Applications, the two biggest professional conferences associated
  with this field of work. As with the similar OEIS platform, it is anticipated
  that a robust EDTS itself would be cited in several of the journals listed
  above, as well as others.






\subsection*{2.	Describe plans for extending the project via external support, if any. (The USA Office of Research can provide help in identifying possible sources of external support relative to this proposal.)}

As the completion of EDTS 1.0 approaches, external funding opportunities
will be proposed by the investigator and his collaborators for the continued
development of the EDTS platform. The National Science Foundation
Division of Advanced Cyberinfrastructure has several opportunities for
supporting such initiatives, including
Computational and Data-Enabled Science and Engineering in Mathematical and Statistical Sciences, and
Cyberinfrastructure for Emerging Science and Engineering Research.
Potential partnerships with academic journals of mathematics will also be explored.














\section*{D.	Facilities and Other Resources:}

\textit{List the facilities available for this project, departmental contributions, and other support for the project not listed in the budget and budget justification pages.}


The investigator has been provided a modern desktop Unix-based computer
for his daily work as a professor. This computer is suitable for
software development and will be the primary instrument used in contributing
to this project.

















\section*{E.	List outcomes and dates of USAFDC grants received in past five years.}

\textit{Provide titles and dates.  List date and title of USAFDC Spring Forum poster presentation.}

N/A













\section*{F.	Describe how the proposed project differs from any previous USAFDC-funded project.}


N/A














\section*{G. List of references cited in the grant application.}

\bibliography{../bib}



% BIOGRAPHICAL SKETCH
% Do not attach a resume. Use the guidelines at:  http://www.southalabama.edu/departments/eforms/graduateschool/biosketch.docx



\end{document}
