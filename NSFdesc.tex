%!TEX root = NSFmaster.tex
%%%%%%%%% PROPOSAL -- 15 pages (including Prior NSF Support)

\required{Project Description}

% From the NSF Grants Proposal Guide:
% "The Project Description should provide a clear statement of the work
% to be undertaken and must include: objectives for the period of the proposed
% work and expected significance; relation to longer-term goals of the PI's
% project; and relation to the present state of knowledge in the field,
% to work in progress by the PI under other support and to work in progress
% elsewhere."

% In my paper \cite{paper01} written during the period of my present
% grant ...

  \section{Overview}

  In the summers of 1967 and 1968, Lynn Arthur Steen and
  J. Arthur Seebach, Jr. coordinated two NSF-funded undergraduate
  research experiences at Saint Olaf College. Each summer, five undergraduate
  students assisted them in investigating and cataloging known and novel
  results in general topology. Specifically, their work was a treatment
  of topological spaces, properties preserved by homeomorphism,
  and theorems connecting these properties (e.g. all compact Hausdorff
  spaces are normal). In 1970, Steen and Seebach published a handbook
  covering this work, titled \textit{Counterexamples in Topology}
  \cite{MR1382863}.

  Mary Ellen Rudin wrote the following in her review \cite{MR1536430}
  of the original text.
  ``\textit{Counterexamples in Topology} is a valuable addition to the small
  collection of books I keep on the shelf in my office.'' ``The book is
  completely unique; no other book now in print serves its purpose.''
  Recognizing the maze of counterexamples littered throughout the
  field of set-theoretic topology, Rudin suggested that students could
  benefit from the text as a guidebook.
  ``Even those of us who work exactly in the area will profit from
  its organization.'' Several other ``\textit{Counterexamples}'' texts
  in the tradition of Stein and Seebach have been published in several fields,
  including (but not limited to) real analysis \cite{MR1256489}, differential
  equations \cite{MR1113487}, probability \cite{MR930671}, and
  graph theory \cite{MR0491272}.

  Of the students involved in these topology REUs, several moved on to careers
  in mathematical research. These include John Feroe,
  Professor of Mathematics and Statistics at Vassar College;
  Gary Gruenhage, Professor Emeritus of Mathematics at Auburn University,
  specializing in set-theoretic topology; and Linda Ness,
  Chief	Research Scientist:	Applied	Research,	Vencore	Labs. Stein and
  Seebach write the following in their introduction to \textit{Counterexamples},
  ``We acknowledge that theirs was a twofold contribution: not only did
  they explore and develop many examples, but they proved by their own
  example the efficacy of examples for the undergraduate study of topology.''

  The handbook stands as a useful resource, even today.
  However, the examples in this original work are relatively elementary,
  and the research community has shifted focus
  away from some topics and onto others over the last four decades.
  To this end,
  the proposed REU program is a next-generation approach to extending
  \textit{Counterexamples}, keeping in mind the needs of
  modern students and researchers in topology, and taking advantage of modern
  technology.

  \{TODO: give overview of PiBase\}

  Participants in the propsoed REU will contribute to the PiBase database, auditing
  its current entries as well as adding new content from more recent
  publications such as Watson's survey of topological planks and
  resolutions \cite{MR1229141}. Participants will also be given mentorship
  as they choose a problem based on this work for the purpose of original
  research.

  As a result of the proposed program, all students of set-theoretic topology
  will benefit, especially the program's participants. General topology
  is the backbone of many mathematical fields, and participants will be
  given the opportunity to develop their knowledge of this core.
  Generally, it is difficult to develop novel open questions in set-theoretic
  topology which are accessible by an undergraduate researcher. However,
  the PiBase application automatically detects unknown properties of
  the spaces within its database, providing a plethora of material
  on which the REU participants, as well as any undergraduate student, may
  work as original research. By uncovering the questions which researchers
  have not yet thought to ask or rigorously pursue, a robust PiBase
  will allow students to obtain valuable experience working on truely
  open problems, while contributing to the collective knowledge of the
  set-theoretic topology research community.

  The proposers' aim is to simultaneously provide an authentic research
  experience for the undergraduate participants while developing the
  PiBase research assistant for use by the entire research
  community. Students are not all expected to pursue
  mathematical research as a career; however, the problem-solving skills
  developed during the program, and the exposure to software
  development and research cyberinfrastructure, will certainly benefit all
  participants regardless of their eventual careers. However, it is
  the investigators' hope that many participants will have their interest
  in mathematical research solidified by this experience, or even have it
  germinated for the first time.

  Researchers of set-theoretic topology will also appreciate the product
  of this REU. Several major open questions in topology simply ask for
  the existence of, or counterexample to, a topological space satisfying
  certain properties (perhaps under various set-theoretic axioms).
  Additionally, it is not uncommon for seminar talks to be derailed by
  pondering the existence of one counterexample or another. So much
  of the community's knowledge is scattered across
  a diaspora of peer-reviwed papers in numerous journals, meaning many
  ``open'' questions may actually be a simple corollary of results from
  two or more heretofore unconnected articles. Likewise, several properties
  have been
  studied under various names, whether for historical reasons, or because
  these properties were later shown to be equivalent; other properties share
  the same names, while actually being distinct (at least in a sufficiently
  general setting). Spaces and properties in the PiBase are tagged with
  unique IDs, preventing any ambiguity, and providing researchers a common
  language when referencing existing spaces and properties from the literature.

  The benefits of this program will not be restricted to only students of
  topology, or even researchers in set-theoretic topology. After the
  PiBase database has been updated to reflect the modern status of topological
  research, data on its utility as a tool for students and researchers may
  be collected. At its core, PiBase is a tool which may be generalized
  to relate the objects, categorical invariants, and theorems relating those
  invariants within any given mathematical category. Once the PiBase is
  battled-tested within one field, it will serve as a proof of concept for
  researchers of different categories, and can be adapted to serve those
  communities as well.

  As described in more detail below, the proposed program will take place
  over two summers, each summer lasting eight weeks
  with a team of eight talented undergraduate
  students. The sponsoring institution is The University of South
  Alabama, located in Mobile, Alabama. The program will be run virtually,
  in order to reach out to all United States citizens, nationals, and
  permanent residents, regardless of their geographical location,
  as well as allow for the seamless interaction of guest speakers and mentors
  from around the world. The resources saved by not funding housing or
  travel for the REU will be repurposed for supporting stipends,
  conference travel, and conference lodging for students.

  Leadership of the proposed program is divided between Dr. Steven Clontz,
  handling the logistics of running the program and the combined
  research and cyberinfrastructure
  expertise required for nurturing a next-generation \textit{Counterexamples},
  and Dr. Ziqin Feng, bringing his research expertise in the fields of set-theroetic
  topology and set theory. They will also be assisted by James Dabbs, the
  developer and maintainer of the PiBase project, who received his Masters
  in mathematics studing set-theoretic topology before entering industry as
  a software engineer.

  \{TODO Add participating organizations'
  commitment to the REU activity.\}

\section{Nature of Student Activities}

  Despite the differing nature of onsite and virtual undergraduate research
  experiences, student activities for the proposed program will be very similar
  to those in an onsite mathematics REU. All meetings will take place
  between organizers and participants using the low-cost persistent chatroom
  and videoconferencing service Slack for Education, utilizing features such
  as screensharing, virtual whiteboarding, and mathematical typesetting.
  Virtual meetings will be scheduled between 8am and 5pm for all
  participating timezones.

  The first two weeks of the program will serve to orient participants
  to the field, building upon their existing knowledge of the topology of
  Euclidean space \(\mathbb R^n\) and any previous topology courses they
  have taken. This orientation will begin by covering selected
  spaces and their properites from the PiBase database,
  and later expanding to theorems in the database. Group investigations will
  take place during morning meetings, led by one of the organizers, followed
  by assigned work to be done in pairs by the participants in the afternoons.
  The length of this orientation period may be adjusted based upon the
  competency of the participants, but particularly in the first summer this
  will also serve to audit the existing contents of the PiBase database.
  Organizers will serve as mentors during this period, and will check in
  with the participants during the afternoon to provide advice or assistance
  as required.

  Following this introductory period, the remainder of the summer will
  consist of each student being assigned several spaces and properties
  from the literature,
  to be studied and catalogued in the PiBase database. These assignments
  will be chosen by the organizers
  based upon the competencies and interests of the participants, and some
  will be assigned to multiple participants for the purposes of peer review
  and collaboration. Once or twice each day,
  a student will lead a presentation with an
  organizer and the other participants based upon the material they have
  investigated since their last presentation. Occassionally, these presentations
  will be substituted with talks from active researchers in set-theoretic topology
  or other fields of interest to the participants. These speakers will
  be brought in virtually through free
  videoconferencing solutions from anywhere in the world.

  As participants work through their assigned material, several questions will
  arise from blanks in the database which cannot be directly
  answered from the literature.
  Under the organizers' mentorship, participants will choose several questions
  for their original research. Generally, these questions will be of one of the
  following forms:
  \begin{itemize}
    \item ``Does space \(S\) have property \(P\)?''
    \item ``Does property \(P\) imply \(Q\), or does there exist a counterexample?''
  \end{itemize}
  Such questions can be automatically generated by PiBase; however, it is up
  to the participants and their mentors to choose questions appropriate for
  their competencies and interests. The expectation for each participant
  is that his or her work will result in one or more publications within
  journals appropriate for the significance of the obtained results. Each
  participant is also expected to give a talk on their research
  at either a meeting of the AMS/MAA, a conference on undergraduate research,
  or a topology conference, following the conclusion of the program.

  \{TODO add note on the significance of the research area\}

  In addition to the mathematical training and experience the participants will
  receive, they will also be exposed to several technologies and topics outside
  set-theoretic topology. These topics include, but are not limited to,
  the cyberinfrastructure of mathematical research, mathematical typesetting,
  web application development, teamwork/leadership skills, pedagogy,
  and presentation skills. These skills will benefit the student participants
  in nearly any chosen career.

  While the development of PiBase as a tool for modern students and researchers
  of topology is a major goal, the primary focus of the proposed program
  is the academic and professional development of the participating students.
  These participants will develop relationships not only with their peers
  and the program organizers, but also with various invited speakers and
  the greater community developing the PiBase source and contributing to its
  database. Participants will be
  required to show evidence of collaboration with
  each other and their mentors on their assigned projects in order
  to develop these important collegial relationships. Participants will be
  encouraged to stay involved in the PiBase community after the completion
  of the REU, perhaps even returning as a virtual speaker during a future year,
  as long as their research interests overlap with the content of the program.

  \{TODO Ziqin Add more details specific to the research\}

\section{The Research Environment}

  \{TODO Ziqin\}

  \{TODO Steven add research bio\}

\section{Student Recruitment and Selection}

  Recruitment for this program will be centralized on a webpage hosted
  by the University of South Alabama. Participants will be able to apply
  through a responsive online form a desktop computer, tablet, or mobile device.
  The site will include a brief description of the program, describing the
  PiBase application and the mathematical concepts involved.
  Emails advertising this site will be disseminated to the organizers'
  network of colleagues in various areas of
  topology, attracting promising candidates from across
  the country with an interest in topology and the cyberinfrastructre of
  mathematical research. The program will also be announced on popular social
  media outlets such as Reddit. Demographics underrepresented in mathematics
  will be especially encouraged to apply; as such, email advertisements
  will be also be distributed to organizations which represent such groups.

  Each application will outline the candidate's academic
  experience and ability, as well as his or her background and interests.
  Only U.S. citizens, U.S. nationals, and
  permanent residents of the United States will be eligible to apply.
  After an evaluation of applications, the organizers will select roughly twice
  as many candidates as there are spaces for the REU as a shortlist. These
  candidates will be given a short interview over Skype or Google Hangouts
  to assess their ability to communicate and collaborate virtually,
  and confirm the academic experience
  and ability represented by their application. These candidates will be
  ranked based on their applications and interviews, and invitations will
  be given to the top half of candidates, with the other candidates put
  on a waitlist in case a selected candidate declines to accept his or her
  invitation. Candidates who can contribute to the diversity of the
  REU program will be given preference when selecting candidates for
  interviews and invitations, as will candidates who represent academic
  institutions where STEM research opportunities are limited. Preference
  will also be given to candidates who demonstrate a need for a virtual
  research experience over a traditional onsite REU.
  Students from the organizers' home institutions
  will be given no preferential treatment during this process, and no more
  than one-quarter of the invited participants will come from those
  institutions.

\section{Project Evaluation and Reporting}

  The two main metrics of success for this project will be the academic success
  of its participants, followed by the utility of the PiBase database they help
  develop during the program. Participants will be encouraged to stay involved
  in the PiBase community after the completion of the program, and to maintain
  correspondence with the program organizers. By the nature of the Slack
  service, most written communictions are catalogued and may be searched for
  evaluation purposes.

  The project directors will collect evaluations from the particpants at the end
  of each summer, and track their publications, talks, and
  accomplishments following the program. Once PiBase is ready for production,
  researchers associated with the Spring Topology and Dynamics Conference
  and the Summer Conference on Topology and its Applications
  will be encouraged to try it and provide feedback on its utility. From this
  combined data, the project directors will write a final evaluation of the
  program for the NSF.

\section{Broader Impacts}

  The broader goal of the proposed program is to develop next-generation
  techniques in the cyberinfrastructure of mathematical research,
  not only within participating students but for the entire research
  community. Generally, researchers in set-theoretic topology
  and its related fields have not
  taken much advantage of technology beyond mathematical typesetting tools
  and email. In contrast, this program relies heavily on vairous internet-based
  communication tools, which if adopted by the wider
  reserach community would enhance productivity in producing new results
  by connecting researchers and facilitating collaboration. It also serves to
  further develop the PiBase topology research assistant, which will help all
  reserachers of topology be more productive in their work by faciliting
  communications, centralizing topological results, and providing a common
  language for otherwise ambiguous terminology or notation.

  By virtue of being a virtual REU, this program also has the potential to
  bring a rigorous research experience to students who otherwise may not
  be able to participate in a traditional onsite REU. The only requirement
  for the participants is consistent access to high-speed interent on a
  laptop or desktop computer; all required software and services will
  be freely available to the participants. Furthermore, because students
  will never ``leave'' the REU host, the research experience can extend
  well beyond the eight weeks for an interested participant, and the
  relationships forged during the REU experience will not waver due to
  geographical distance after the end of the program.

\section{Results From Prior NSF Support}

  No prior NSF support has been given for this project.
