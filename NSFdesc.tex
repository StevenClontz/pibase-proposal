%!TEX root = NSFmaster.tex
%%%%%%%%% PROPOSAL -- 15 pages (including Prior NSF Support)

\required{Project Description}

% From the NSF Grants Proposal Guide:
% "The Project Description should provide a clear statement of the work
% to be undertaken and must include: objectives for the period of the proposed
% work and expected significance; relation to longer-term goals of the PI's
% project; and relation to the present state of knowledge in the field,
% to work in progress by the PI under other support and to work in progress
% elsewhere."

% In my paper \cite{paper01} written during the period of my present
% grant ...

  \section{Overview}

  In the summers of 1967 and 1968, Lynn Arthur Steen and
  J. Arthur Seebach, Jr. coordinated two NSF-funded undergraduate
  research experiences at Saint Olaf College. Each summer, five undergraduate
  students assisted them in investigating and cataloging known and novel
  results in general topology. Specifically, their work was a treatment
  of topological spaces, properties preserved by homeomorphism,
  and theorems connecting these properties (e.g. all compact Hausdorff
  spaces are normal). In 1970, Steen and Seebach published a handbook
  covering this work, titled \textit{Counterexamples in Topology}
  \cite{MR1382863}.

  Mary Ellen Rudin wrote the following in her review \cite{MR1536430}
  of the original text.
  ``\textit{Counterexamples in Topology} is a valuable addition to the small
  collection of books I keep on the shelf in my office.'' ``The book is
  completely unique; no other book now in print serves its purpose.''
  Recognizing the maze of counterexamples littered throughout the
  field of set-theoretic topology, Rudin suggested that students could
  benefit from the text as a guidebook.
  ``Even those of us who work exactly in the area will profit from
  its organization.'' Several other ``\textit{Counterexamples}'' texts
  in the tradition of Stein and Seebach have been published in several fields,
  including (but not limited to) real analysis \cite{MR1256489}, differential
  equations \cite{MR1113487}, probability \cite{MR930671}, and
  graph theory \cite{MR0491272}.

  Of the students involved in these topology REUs, several moved on to careers
  in mathematical research. These include John Feroe,
  Professor of Mathematics and Statistics at Vassar College;
  Gary Gruenhage, Professor Emeritus of Mathematics at Auburn University,
  specializing in set-theoretic topology; and Linda Ness,
  Chief	Research Scientist:	Applied	Research,	Vencore	Labs. Stein and
  Seebach write the following in their introduction to \textit{Counterexamples},
  ``We acknowledge that theirs was a twofold contribution: not only did
  they explore and develop many examples, but they proved by their own
  example the efficacy of examples for the undergraduate study of topology.''

  The handbook stands as a useful resource, even today.
  However, the examples in this original work are relatively elementary,
  and the research community has shifted focus
  away from some topics and onto others over the last four decades.
  To this end,
  the proposed REU program is a next-generation approach to extending
  \textit{Counterexamples}, keeping in mind the needs of
  modern students and researchers in topology, and taking advantage of modern
  technology.

  \{TODO: give overview of PiBase\}

  Participants in the propsoed REU will contribute to the PiBase database, auditing
  its current entries as well as adding new content from more recent
  publications such as Watson's survey of topological planks and
  resolutions \cite{MR1229141}. Participants will also be given mentorship
  as they choose a problem based on this work for the purpose of original
  research.

  As a result of the proposed program, all students of set-theoretic topology
  will benefit, especially the program's participants. General topology
  is the backbone of many mathematical fields, and participants will be
  given the opportunity to develop their knowledge of this core.
  Generally, it is difficult to develop novel open questions in set-theoretic
  topology which are accessible by an undergraduate researcher. However,
  the PiBase application automatically detects unknown properties of
  the spaces within its database, providing a plethora of material
  on which the REU participants, as well as any undergraduate student, may
  work as original research. By uncovering the questions which researchers
  have not yet thought to ask or rigorously pursue, a robust PiBase
  will allow students to obtain valuable experience working on truely
  open problems, while contributing to the collective knowledge of the
  set-theoretic topology research community.

  The proposers' aim is to simultaneously provide an authentic research
  experience for the undergraduate participants, adjusting for the length of
  the program, the inexperience of the participants, and the goal of
  developing the PiBase database. Students are not all expected to pursue
  mathematical research as a career; however, the problem-solving skills
  developed during the program, and the exposure to software
  development and research cyberinfrastructure, will certainly benefit all
  participants regardless of their eventual careers. However, it is
  the investigators' hope that many participants will have their interest
  in mathematical research solidified by this experience, or even have it
  germinated for the first time.

  Researchers of set-theoretic topology will also appreciate the product
  of this REU. Several major open questions in topology simply ask for
  the existence of, or counterexample to, a topological space satisfying
  certain properties (perhaps under various set-theoretic axioms).
  Additionally, it is not uncommon for seminar talks to be derailed by
  pondering the existence of one counterexample or another. So much
  of the community's knowledge is scattered across
  a diaspora of peer-reviwed papers in numerous journals, meaning many
  ``open'' questions may actually be a simple corollary of results from
  two or more heretofore unconnected articles. Likewise, several properties
  have been
  studied under various names, whether for historical reasons, or because
  these properties were later shown to be equivalent; other properties share
  the same names, while actually being distinct (at least in a sufficiently
  general setting). Spaces and properties in the PiBase are tagged with
  unique IDs, preventing any ambiguity, and providing researchers a common
  language when referencing existing spaces and properties from the literature.

  The benefits of this program will not be restricted to only students of
  topology, or even researchers in set-theoretic topology. After the
  PiBase database has been updated to reflect the modern status of topological
  research, data on its utility as a tool for students and researchers may
  be collected. At its core, PiBase is a tool which may be generalized
  to relate the objects, categorical invariants, and theorems relating those
  invariants within any given mathematical category. Once the PiBase is
  battled-tested within one field, it will serve as a proof of concept for
  researchers of different categories, and can be adapted to serve those
  communities as well.

  As described in more detail below, the proposed program will take place
  over two summers, each summer lasting eight weeks
  with a team of six talented undergraduate
  students. The first summer will take place at the University of South
  Alabama, located in Mobile, Alabama, and targeting participants from
  the southeastern United States. The second summer will be held
  virtually, without a bias towards any students in a particular geographical
  region of the United States.

  Leadership of the proposed program is divided between Dr. Steven Clontz,
  handling the logistics of running the program and the combined
  research and cyberinfrastructure
  expertise required for nurturing a next-generation \textit{Counterexamples},
  and Dr. Ziqin Feng, bringing his research expertise in the fields of set-theroetic
  topology and set theory. They will be assisted by James Dabbs, the
  developer and maintainer of the PiBase project, who received his Masters
  in mathematics studing set-theoretic topology before entering industry as
  a software engineer. The first summer will bring these players together
  with the first batch of student participants onsite, to build the foundation
  upon which future summers will be run virtually. By running the second
  summer virtually, a more diverse audience of potential participants may
  be reached, as well as opening up participation to potential speakers and
  mentors across the country. In addition, participants will be better equipped
  to continue contributions
  to the PiBase database after the end of the program as their environment
  will not change, and the costs of running the program will be reduced.

  \{TODO Add participating organizations'
  commitment to the REU activity.\}

\section{Nature of Student Activities}

Despite the differing nature of onsite and virtual undergraduate research
experiences, student activities will remain the same between the first summer
and the second. In the second summer, all meetings will take place
between organizers and participants using free videoconferencing services
such as Google Hangouts, utilizing screensharing alongside virtual whiteboarding
and mathematical typesetting programs in order to share sketched and written
mathematical concepts. All virtual meetings will be scheduled between
9am Pacific and 6pm Eastern to accommodate varying timezones of the participants.

The first two weeks of the program will serve to orient participants
to the field, building upon their existing knowledge of the topology of
Euclidean space \(\mathbb R^n\) and any previous topology courses they
have taken. This orientation will begin by covering selected
spaces and their properites from the PiBase database,
and later expanding to theorems in the database. Group investigations will
take place during morning meetings, led by one of the organizers, followed
by assigned work to be done in pairs by the participants in the afternoons.
The length of this orientation period may be adjusted based upon the
competency of the participants, but particularly in the first summer this
will also serve to audit the existing contents of the PiBase database.
Organizers will serve as mentors during this period, and will check in
with the participants during the afternoon to provide advice or assistance
as required.

Following this introductory period, the remainder of the summer will
consist of each student being assigned several spaces and properties
from the literature,
to be studied and catalogued in the PiBase database. These assignments
will be chosen by the organizers
based upon the competencies and interests of the participants, and some
will be assigned to multiple participants for the purposes of peer review
and collaboration. Once or twice each day,
a student will lead a presentation with an
organizer and the other participants based upon the material they have
investigated since their last presentation. Occassionally, these presentations
will be substituted with talks from active researchers in set-theoretic topology
or other fields of interest to the participants. These speakers may be in
person during the onsite summer, or be brought in virtually through free
videoconferencing solutions from anywhere in the world.

As participants work through their assigned material, several questions will
arise from blanks in the database which cannot be directly
answered from the literature.
Under the organizers' mentorship, participants will choose several questions
for their original research. Generally, these questions will be of one of the
following forms:
\begin{itemize}
  \item ``Does space \(S\) have property \(P\)?''
  \item ``Does property \(P\) imply \(Q\), or does there exist a counterexample?''
\end{itemize}
Such questions can be automatically generated by PiBase; however, it is up
to the participants and their mentors to choose questions appropriate for
their competencies and interests. The expectation for each participant
is that his or her work will result in one or more publications within
journals appropriate for the significance of the obtained results. Each
participant is also expected to give a talk on their research
at either a meeting of the AMS/MAA, a conference on undergraduate research,
or a topology conference, following the conclusion of the program.

\{TODO add note on the significance of the research area\}

In addition to the mathematical training and experience the participants will
receive, they will also be exposed to several technologies and topics outside
set-theoretic topology. These topics include, but are not limited to,
the cyberinfrastructure of mathematical research, mathematical typesetting,
web application development, teamwork/leadership skills, pedagogy,
and presentation skills. These skills will benefit the student participants
in nearly any chosen career.

While the development of PiBase as a tool for modern students and researchers
of topology is a major goal, the primary focus of the proposed program
is the academic and professional development of the participating students.
These participants will develop relationships not only with their peers
and the program organizers, but also with various invited speakers and
the greater community developing the PiBase source and contributing to its
database. During both the onsite and virtual summers, participants will be
required to show evidence of collaboration with
each other and their mentors on their assigned projects in order
to develop these important collegial relationships. Participants will be
encouraged to stay involved in the PiBase community after the completion
of the REU, perhaps even returning as a virtual speaker during a future year,
as long as their research interests overlap with the content of the program.

In order to foster commraderie amongst the participants, games and other
social events will be organized or offered, particularly during the onsite year.
Some, especially those mathematical in nature, will be required and take place
during the program itself. Examples of such activities include working on
puzzles provided by Mathematical Puzzle Programs and other online resources,
and playing and studying games with connections to mathematics such as Nim.
During the onsite year, participants will have the opportunity to visit
the beach at Gulf Shores, \{TODO add other fun social stuff here, or remove
paragraph if inappropriate?\}

\{TODO Ziqin Add more details specific to the research\}

\section{The Research Environment}

\{TODO Ziqin\}

\{TODO Steven add research bio\}

\section{Student Recruitment and Selection}

Recruitment for this program will be centralized on a webpage hosted
by the University of South Alabama. Participants will be able to apply
through a responsive online form a desktop computer, tablet, or mobile device.
The site will include a brief description of the program, describing the
PiBase application and the matheatical concepts involved.
Emails advertising this site will be disseminated to every mathematics
and/or computer science department of every four-year college and
university in Alabama. Other major schools will also be notified within
adjoining states, expanding outward geographically from South Alabama.
In addition, colleagues of the organizers in the area of set-theoretic
topology will be encouraged to promote the opportunity to their departments
and students as well. In particular, the organizers will rely on this
network of fellow researchers to attract promising candidates from across
the country with an interest in topology and the cyberinfrastructre of
mathematical research. Demographics underrepresented in mathematics
will be especially encouraged to apply.

% How do we actually go about the selection? First, we vaguely intend to have 4 female participants, and at least 4 participants “from” the Southeast, although, as indicated
% somewhere above, it is a bit tricky to judge just how Southeastern a participant is. For example, in 2011 we had Nicole Looper from northern Florida (which is culturally about as Southeastern as Alabama and Georgia), but her home institution was Dartmouth College.
% In any case, we go over the likely candidates who are female and/or Southeastern first, and tender offers to any of those who seem sure bets. We are also on the lookout for African- American or Hispanic applicants, and these will be included in the first sweep, but, frankly, there aren’t many such applicants. After the first run-through of the candidates in the favored categories, we have a few offers out, and we have a list of alternatives. At this point we turn our attention to the large pile of applicants who are male and not Southeastern.
% In the selection process we pay special attention to the “math-biography” that is required in the application. The applicants are to describe their experiences with math., including bad ones, what they are interested in, how they got interested—whatever they feel like saying.
% We think that we can tell from these math-bios whether or not the applicant will be a good fit for our program—it really has to do with whether there is some sign that the applicant might really be interested in mathematics, in the way that someone would have to be interested in order to do research in it. Unfortunately, this way of examining the math-bio probably discriminates against the unsophisticated, those who really don’t know what they think about math, and envision an REU as some sort of summer mathematics course, only with a “project” in place of a final test. Maybe it’s not unfortunate for those people to be discriminated against—maybe they are better off not going into our REU.

\{The overall quality of the student recruitment and selection processes and criteria will be an important element in the evaluation of the proposal. The recruitment plan should be described with as much specificity as possible, including the types and/or names of academic institutions where students will be recruited and the efforts that will be made to attract members of underrepresented groups (women, minorities, and persons with disabilities).
A significant fraction of the student participants at an REU Site must come from outside the host institution or organization, and at least half of the student participants must be recruited from academic institutions where research opportunities in STEM are limited (including two-year colleges). The number of students per project should be appropriate to the institutional or organizational setting and to the manner in which research is conducted in the discipline. (The typical REU Site hosts 8-10 students per year.) Proposals involving fewer than six students per year are discouraged.

Undergraduate student participants supported with NSF funds in either
REU Sites or REU Supplements must be U.S. citizens, U.S. nationals, or
permanent residents of the United States.\}

\section{Project Evaluation and Reporting}

\section{Broader Impacts}

\section{Results From Prior NSF Support}

No prior NSF support has been given for this project.
