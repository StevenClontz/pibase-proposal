%!TEX root = NSFmaster.tex
%%%%%%%%% PROPOSAL -- 15 pages (including Prior NSF Support)

\required{Project Description}

% From the NSF Grants Proposal Guide:
% "The Project Description should provide a clear statement of the work
% to be undertaken and must include: objectives for the period of the proposed
% work and expected significance; relation to longer-term goals of the PI's
% project; and relation to the present state of knowledge in the field,
% to work in progress by the PI under other support and to work in progress
% elsewhere."

% In my paper \cite{paper01} written during the period of my present
% grant ...

  \section{Overview}

  In the summers of 1967 and 1968, Lynn Arthur Steen and
  J. Arthur Seebach, Jr. coordinated two NSF-funded undergraduate
  research experiences at Saint Olaf College. Each summer, five undergraduate
  students assisted them in investigating and cataloging known and novel
  results in general topology. Specifically, their work was a treatment
  of topological spaces, properties preserved by homeomorphism,
  and theorems connecting these properties (e.g. all compact Hausdorff
  spaces are normal). In 1970, Steen and Seebach published a handbook
  covering this work, titled \textit{Counterexamples in Topology}
  \cite{MR1382863}.

  Mary Ellen Rudin wrote the following in her review \cite{MR1536430}
  of the original text.
  ``\textit{Counterexamples in Topology} is a valuable addition to the small
  collection of books I keep on the shelf in my office.'' ``The book is
  completely unique; no other book now in print serves its purpose.''
  Recognizing the maze of counterexamples littered throughout the
  field of set-theoretic topology, Rudin suggested that students could
  benefit from the text as a guidebook.
  ``Even those of us who work exactly in the area will profit from
  its organization.'' Several other ``\textit{Counterexamples}'' texts
  in the tradition of Stein and Seebach have been published in several fields,
  including (but not limited to) real analysis \cite{MR1256489}, differential
  equations \cite{MR1113487}, probability \cite{MR930671}, and
  graph theory \cite{MR0491272}.

  Of the students involved in these topology REUs, several moved on to careers
  in mathematical research. These include John Feroe,
  Professor of Mathematics and Statistics at Vassar College;
  Gary Gruenhage, Professor Emeritus of Mathematics at Auburn University,
  specializing in set-theoretic topology; and Linda Ness,
  Chief	Research Scientist:	Applied	Research,	Vencore	Labs. Stein and
  Seebach write the following in their introduction to \textit{Counterexamples},
  ``We acknowledge that theirs was a twofold contribution: not only did
  they explore and develop many examples, but they proved by their own
  example the efficacy of examples for the undergraduate study of topology.''

  The handbook stands as a useful resource, even today.
  However, the examples in this original work are relatively elementary,
  and the research community has shifted focus
  away from some topics and onto others over the last four decades.
  To this end,
  the proposed REU program is a next-generation approach to extending
  \textit{Counterexamples}, keeping in mind the needs of
  modern students and researchers in topology, and taking advantage of modern
  technology.

  \{TODO: give overview of piBase\}

  Participants in the propsoed REU will contribute to the piBase database, auditing
  its current entries as well as adding new content from more recent
  publications such as Watson's survey of topological planks and
  resolutions \cite{MR1229141}. Participants will also be given mentorship
  as they choose a problem based on this work for the purpose of original
  research.

  As a result of the proposed program, all students of set-theoretic topology
  will benefit, especially the program's participants. General topology
  is the backbone of many mathematical fields, and participants will be
  given the opportunity to develop their knowledge of this core.
  Generally, it is difficult to develop novel open questions in set-theoretic
  topology which are accessible by an undergraduate researcher. However,
  the piBase application automatically detects unknown properties of
  the spaces within its database, providing a plethora of material
  on which the REU participants, as well as any undergraduate student, may
  work as original research. By uncovering the questions which researchers
  have not yet thought to ask or rigorously pursue, a robust piBase
  will allow students to obtain valuable experience working on truely
  open problems, while contributing to the collective knowledge of the
  set-theoretic topology research community.

  The proposers' aim is to simultaneously provide an authentic research
  experience for the undergraduate participants, adjusting for the length of
  the program, the inexperience of the participants, and the goal of
  developing the piBase database. Students are not all expected to pursue
  mathematical research as a career; however, the problem-solving skills
  developed during the program, and the exposure to software
  development and research cyberinfrastructure, will certainly benefit all
  participants regardless of their eventual careers. However, it is
  the investigators' hope that many participants will have their interest
  in mathematical research solidified by this experience, or even have it
  germinated for the first time.

  Researchers of set-theoretic topology will also appreciate the product
  of this REU. Several major open questions in topology simply ask for
  the existence of, or counterexample to, a topological space satisfying
  certain properties (perhaps under various set-theoretic axioms).
  Additionally, it is not uncommon for seminar talks to be derailed by
  pondering the existence of one counterexample or another. So much
  of the community's knowledge is scattered across
  a diaspora of peer-reviwed papers in numerous journals, meaning many
  ``open'' questions may actually be a simple corollary of results from
  two or more heretofore unconnected articles. Likewise, several properties
  have been
  studied under various names, whether for historical reasons, or because
  these properties were later shown to be equivalent; other properties share
  the same names, while actually being distinct (at least in a sufficiently
  general setting). Spaces and properties in the piBase are tagged with
  unique IDs, preventing any ambiguity, and providing researchers a common
  language when referencing existing spaces and properties from the literature.

  The benefits of this program will not be restricted to only students of
  topology, or even researchers in set-theoretic topology. After the
  piBase database has been updated to reflect the modern status of topological
  research, data on its utility as a tool for students and researchers may
  be collected. At its core, piBase is a tool which may be generalized
  to relate the objects, categorical invariants, and theorems relating those
  invariants within any given mathematical category. Once the piBase is
  battled-tested within one field, it will serve as a proof of concept for
  researchers of different categories, and can be adapted to serve those
  communities as well.

  \{TODO: Provide a brief description of the targeted student participants,
  organizational structure, timetable, and participating organizations'
  commitment to the REU activity.\}

%
% Mathematical research is not like research in the sciences or engineering. We do get participants who expect the mentor/mentee model, in which they are “assigned a project” and then guided through the project by a mentor. While we are willing and able to provide such participants with what they expect and/or need, in the interest of authenticity we feel
% it to be of the utmost importance to introduce all the participants to the true model of mathematical research as an institutional activity, a model which descends to us from the Pythagoreans and the gathering around Euclid at the Alexandrian library: the research institute. At every university where mathematical research is done, the mathematics department, or parts of it, constitute a research institute. How is mathematics conducted in a research institute? In all sorts of ways, in any way imaginable! It’s Liberty Hall! There is no
%   check list, there is no plan—there is a “community of scholars”, equal in opportunity, when it comes to research, regardless of their ages and credentials. These scholars read, write, think, and converse as they please, about whatever interests them, and, somehow, mathematical discoveries are made.
% With unversed undergraduates and only 8 weeks, we have to stimulate somewhat artificially the formation of our community of scholars and proactively get them thinking about mathematical questions; what we actually do is described below under Nature of Student Activities. We think that we score well on authenticity: our regimentation is mild, and the participants all seem to get that they can work on anything they want to work on, or nothing at all—just like in real life! And our participants wind up working in all sorts of ways—in groups of two or more, with or without participation of faculty, sometimes in lone wolf mode (3 of the 5 best results to come out of our program were found by lone wolf participants), as well as in standard mentor/mentee mode with one of the co-directors. This variety of social constructs in the program is a good sign—it is what we are aiming for, a sort of replica of a modern mathematics department that is active in research, and plausibly a worthy descendant of the groups of ancient Greeks that (arguably) set the tone for mathematical research, forever.
% How much of an experience do our participants actually get? Of the nearly 100 participants in our 9 programs, we judge that about half achieved a genuine mathematical discovery. It’s hard to judge, but almost all of the others put out a lot of effort, contributed something (a useful observation, an incompletely formed but relevant idea) to joint research efforts, and were present to admire and grasp the discoveries being made. (We are all about encouragement, not discouragement, and it should be noted that at least 5 of those from our first 9 programs who made no notable discovery during the program went on to graduate school in mathematics, and at least 2 of those have achieved Ph.D.’s.) We think that all of the participants came away with a pretty good idea of the level of effort and concentration necessary to do mathematical research, and some kind of esthetic appreciation of mathematics in the wild.
% The excellent atmosphere that we have enjoyed in all 9 REU’s has been due in no small part to the unselfish and unstinting cooperation and contributions of our colleagues and the graduate students of our department. Auburn University is internationally recognized in several areas of mathematics, most especially in design theory, but also in graph theory, coding theory, algebra, geometry, topology, and analysis. (The latter three areas are not squarely in the focus of our REU, but we have had very enlightening and interesting talks from colleagues and graduate students in those areas.) Our graduate students, our colleagues, and the distinguished visitors that frequently alight at Auburn have been generous in giving talks and posing problems to our participants and, in a few cases, in working with them. We anticipate a continuation of the lively scholarly activity based around the REU program, should funding for it be renewed.
% Regarding “targeting” participants: We have an affirmative action bias toward participants from the Southeast, women, and minorities, roughly in that order, with our greatest bias being toward participants from the Southeast. We take the trouble to send a special recruiting email to mathematics and computer science departments in the Southeast; we usually manage to have half and sometimes more of our participants from the Southeast, and half but sometimes less of our participants female. To be honest, there is ambiguity in the phrase “from the
%   Southeast.” For instance, in 2011 we had one female participant from Connecticut who was a student at the University of Richmond, in Virginia, and another female participant from northern Florida who was a student at Dartmouth. On the other hand, we had another 4 participants that year who were unambiguously from and going to school in the Southeast.
% Often we have no students from Auburn University in the program, but we have had as many as three. In 2012 we had one, in 2011, none, and in 2010, two. In our last 3 programs, 2013, 2014, and 2015, we have had 2, 1, and 2 participants from Auburn, respectively.


\section{Nature of Student Activities}

\section{The Research Environment}

\section{Student Recruitment and Selection}

\section{Project Evaluation and Reporting}

\section{Broader Impacts}

\section{Results From Prior NSF Support}

No prior NSF support has been given for this project.
